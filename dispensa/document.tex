\documentclass[a4paper,12pt]{article} % Prepara un documento per carta A4, con un font di dimensione 12pt

\usepackage[french,italian]{babel} % Adatta LaTeX alle convenzioni tipografiche italiane,
% e ridefinisce alcuni titoli in italiano, come "Capitolo" al posto di "Chapter",
% se il vostro documento è in italiano
% l'opzione  linguistica 'french' è necessaria per l'abilitazione della
% successiva istruzione <<\frenchspacing>> 
\usepackage[T1]{fontenc} % Riga da togliere se si compila con PDFLaTeX
\usepackage[utf8]{inputenc} % Consente l'uso caratteri accentati italiani
\usepackage{verbatim}
\usepackage{guit}
\usepackage[footnote]{acronym}
\frenchspacing % forza LaTeX ad una spaziatura uniforme, invece di lasciare più spazio
% alla fine dei punti fermi come da convenzione inglese: richiede opzione linguistica 'french'

\title {Esempio di documento in \LaTeX} % \LaTeX è una macro che compone il logo "LaTeX"
% I commenti (introdotti da %) vengono ignorati

\author{Mario Rossi}
\date {8 aprile 2002}
% in alternativa a \date il comando \today introduce la data di sistema.



\begin {document}
\maketitle % Genera il titolo sulle istruzioni  \title, \author e \date

\begin{abstract} % Questo è l'inizio dell'ambiente "abstract".
	% L'ambiente abstract è fatto per contenere un riassunto del contenuto.
	Breve dimostrazione dell'uso di \LaTeX.
\end{abstract} % Qui termina l'ambiente ''abstract''

\tableofcontents % Prepara l'indice generale

\section{Testo normale} %
è possibile scrivere il testo dell'articolo normalmente, ed \emph{enfatizzare} alcune parti del discorso. %
Una riga vuota nel testo indica la fine di un paragrafo.


Il simbolo percentuale "\%" permette di aggiungere commenti al codice senza che siano visibili nel documento, per invece renderlo visibile nel documento basta farlo precedere dal carattere backslash.
\\ \LaTeX considera commento tutto ciò che segue il simbolo \% sulla medesima riga
\\ esempio. il 22\% di 100 è 22, mentre il 50\% è 50. % commento il 70% è 70
% esempio 
%

Il simbolo percentuale (\%) non è l'unico modo per fare commenti in \LaTeX, infatti è possibile generare blocchi di commenti su più righe usando l'istruzione begin seguita da {comment} e terminarla con l'istruzione end seguita da {comment}.
\\ esempio. il commento tra questi blocchi non viene poi mostrato nel documento. \begin{comment} codice all'interno dell'ambiente commento
riga commento 1
riga commento 2
riga commento 3
\end{comment} Il codice dopo questa istruzione viene comunque ignorato

\section{Formule} %
La forza di \LaTeX\ sono per le formule, sia in linea (ad esempio \(y=x^2\)) che messe in bella mostra in un'area propria: \[y=\sqrt{x+y}\]

\section{Poesia} %
L'ambiente ``verse'' è usato per comporre tipograficamente le poesie:
\begin{verse}
	La vispa Teresa avea tra l'erbetta\\ % la doppia barra inversa forza a capo
	al volo sorpresa gentil farfalletta.
\end{verse}

\section{Acronimi}
\begin{acronym}   
	\acro{GUIT} [\GuIT] {Gruppo Utilizzatori Italiani di TeX e LaTeX}
	
	{\small È un’associazione che si prefigge di aumentare la diffusione di
		\TeX{} e \LaTeX{} in Italia.\par}
	\acro{WYSIWYM}{What You See Is What You Mean}
	{\small L’acronimo (“ciò che vedi è ciò che intendi”) indica un
		programma di scrittura dotato di una composizione asincrona.\par}
\end{acronym}
Altri Acronimi:
\begin{acronym}  
	\acro{GUIT} [\GuIT] {Gruppo Utilizzatori Italiani di TeX e LaTeX} (ripetizione)	
	\acro{CAP} [CAP] {Codice Avviamento Postale}
	\acro{CAP} {Codice di Sviamento Postale}
	\acro{4WD} {Four-Wheel Drive} 
	\acro{4WD} {Quattro ruote motrici} 
	\acro{ABI} {Associazione Bancaria Italiana} 
	\acro{ABS} {Antilock Braking System} 
	\acro{ABS} {Sistema anti bloccaggio} 
	\acro{ACI} {Automobile Club Italia} 
	\acro{4WD} {Quattro gomme motrici} 
	\acro{4WD} {Quattro gomme motrici} 
	\acro{ETA} {Tempo stimato d'arrivo}
	\acro{MVP} {Most Valuable Player}
	\acro{MVP} {more valuable player}
	\acro{MSC} {Michael Schumacher} %Michael Schumacher
	\acro{PET} {Posta Elettronica Certificata}
	\acro{AMOLED} {Active Matrix Organic Light Emitting Diode}
	\acro{ABS} {Auto Braking System}
	\acro{PM} {Particolare}
	\acro{UNI} {università}
	\acro{DRAM} {Dynamic Random Access Memory}
\end{acronym}

La prima volta che si scrive un acronimo viene stampato il suo nome per esteso, seguito dall’acronimo fra parentesi: \ac{GUIT}, per esempio.
Le volte successive viene stampato solo l’acronimo: \ac{GUIT}.
\begin{acronym}  
	\acro{GUIT} [\GuIT] {Gruppo Utilizzatori di TeX e LaTeX} 	 
\end{acronym} \end{document}